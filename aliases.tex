%!TEX root = ts.tex

\rSec0[parametric_aliases]{Parametric aliases}

Parametric aliases provide a machine-independent mechanism to specify the desired allocation size built-in or extended types.

The following wording changes are relative to N4527.

\rSec1[compliance]{Freestanding implementations}

Add the following entry to table 16.

\begin{addedblock}
\begin{libreqtab3}
    {Macros}
    {tab:macros}
    \\ \topline
    \lhdr{chapter}  &
    \chdr{description}  &
    \rhdr{header} \\ \capsep
    \endfirsthead
    \continuedcaption\\
    \hline
    \lhdr{chapter}  &
    \chdr{description}  &
    \rhdr{header} \\ \capsep
    \endhead
    
    18.4+ & Floating-point types & <cstdfloat>
    \\ \rowsep     
    
\end{libreqtab3} 
\end{addedblock}

\rSec1[support.general]{General}

Add the following entry to table 29.

\begin{addedblock}
    \begin{libreqtab3}
        {Macros}
        {tab:macros}
        \\ \topline
        \lhdr{chapter}  &
        \chdr{description}  &
        \rhdr{header} \\ \capsep
        \endfirsthead
        \continuedcaption\\
        \hline
        \lhdr{chapter}  &
        \chdr{description}  &
        \rhdr{header} \\ \capsep
        \endhead
        
        18.4+ & Floating-point types & <cstdfloat>
        \\ \rowsep     
        
    \end{libreqtab3} 
\end{addedblock}

\rSec1[cstdint.syn]{Header <cstdint>}

Add the following entries to the synopsis before paragraph 1.

\begin{addedblock}
\begin{codeblock}
namespace std {
  template<int bits> alias exact_2int = implementation-defined;
  template<int bits> alias fast_2int = implementation-defined;
  template<int bits> alias least_2int = implementation-defined;
  template<int bits> alias exact_2uint = implementation-defined;
  template<int bits> alias fast_2uint = implementation-defined;
  template<int bits> alias least_2uint = implementation-defined;
}
#define MAX_BITS_2INT implementation-defined;
#define MAX_BITS_2UINT implementation-defined;
\end{codeblock}    
\end{addedblock}

\rSec1[cstdint.parametric]{Parametric types}

Add a new section with the following paragraphs.

\begin{addedblock}
The aliases below are conditionally supported. The macro \tcode{MAX_BITS_2INT} shall give the largest integer size (in bits) supported by the aliases. [Note: All variants support the same sizes. —end note] If these aliases are not supported, the value shall be 0. Any parameter to the alias shall be in the range 1 to \tcode{MAX_BITS_2INT}.

\begin{itemdecl}
template<int bits> alias exact_2int    
\end{itemdecl}

\begin{itemdescr}
The alias \tcode{exact_2int} refers to a built-in signed binary integer type of exactly \tcode{bits} bits. If there are two types of the same size, it refers to the type that is closest to int in promotion order. The type must represent negative values with two's-complement representation.    
\end{itemdescr}

\begin{itemdecl}
template<int bits> alias fast_2int    
\end{itemdecl}

\begin{itemdescr}
The alias \tcode{fast_2int} refers to the fastest built-in signed binary integer type of at least \tcode{bits} bits. If there are two types of the same size, it refers to the type that is closest to int in promotion order. The type must represent negative values with two's-complement representation.    
\end{itemdescr} 

\begin{itemdecl}
template<int bits> alias least_2int    
\end{itemdecl}

\begin{itemdescr}
The alias \tcode{least_2int} refers to the smallest built-in signed binary integer type of at least \tcode{bits} bits. If there are two types of the same size, it refers to the type that is closest to int in promotion order. The type must represent negative values with two's-complement representation.    
\end{itemdescr} 

\begin{itemdecl}
template<int bits> alias exact_2uint    
\end{itemdecl}

\begin{itemdescr}
The alias \tcode{exact_2uint} refers to a built-in unsigned binary integer type of exactly \tcode{bits} bits. If there are two types of the same size, it refers to the type that is closest to unsigned int in promotion order.    
\end{itemdescr} 

\begin{itemdecl}
template<int bits> alias fast_2uint    
\end{itemdecl}

\begin{itemdescr}
The alias \tcode{fast_2uint} refers to the fastest built-in unsigned binary integer type of at least \tcode{bits} bits. If there are two types of the same size, it refers to the type that is closest to unsigned int in promotion order.    
\end{itemdescr} 

\begin{itemdecl}
template<int bits> alias least_2uint    
\end{itemdecl}

\begin{itemdescr}
The alias \tcode{least_2uint} refers to the smallest built-in unsigned binary integer type of at least \tcode{bits} bits. If there are two types of the same size, it refers to the type that is closest to unsigned int in promotion order.    
\end{itemdescr}    
\end{addedblock}

\rSec1[cstdfloat]{Floating-point types}

After section 18.4, add a new section. It has no direct contents.

\rSec1[cstdfloat.syn]{Header <cstdfloat>}

Add a new section.

\begin{addedblock}
    \begin{codeblock}
namespace std {
  template<int bits> alias exact_2ieeefloat = implementation-defined;
  template<int bits> alias fast_2ieeefloat = implementation-defined;
  template<int bits> alias least_2ieeefloat = implementation-defined;
  template<int bits> alias exact_10ieeefloat = implementation-defined;
  template<int bits> alias fast_10ieeefloat = implementation-defined;
  template<int bits> alias least_10ieeefloat = implementation-defined;
}
#define MAX_BITS_2IEEEFLOAT implementation-defined;
#define MAX_BITS_10IEEEFLOAT implementation-defined;
    \end{codeblock}    
\end{addedblock}

\rSec1[cstdfloat.parametric]{Parametric types}

Add a new section with the following paragraphs.

\begin{addedblock}
The aliases below are conditionally supported. The macro \tcode{MAX_BITS_2IEEEFLOAT} shall give the largest binary floating-point size (in bits) supported by the aliases. [Note: All variants support the same sizes. —end note] If none of these aliases are supported, the value shall be 0. The parameter to the alias shall be in the range 1 to \tcode{MAX_BITS_2IEEEFLOAT}.

\begin{itemdecl}
template<int bits> alias exact_2ieeefloat
\end{itemdecl}

\begin{itemdescr}
The alias \tcode{exact_2ieeefloat} refers to a built-in binary floating-point type of exactly \tcode{bits} bits. If there are two types of the same size, it refers to the type that is closest to double in promotion order. The type must use IEEE representation.
\end{itemdescr}

\begin{itemdecl}
template<int bits> alias fast_2ieeefloat
\end{itemdecl}

\begin{itemdescr}
The alias \tcode{fast_2ieeefloat} refers to the fastest built-in binary floating-point type of at least \tcode{bits} bits. If there are two types of the same size, it refers to the type that is closest to double in promotion order. The type must use IEEE representation.
\end{itemdescr}

\begin{itemdecl}
template<int bits> alias least_2ieeefloat
\end{itemdecl}

\begin{itemdescr}
The alias \tcode{least_2ieeefloat} refers to the smallest built-in binary floating-point type of at least \tcode{bits} bits. If there are two types of the same size, it refers to the type that is closest to double in promotion order. The type must use IEEE representation.
\end{itemdescr}

The aliases below are conditionally supported. The macro \tcode{MAX_BITS_10IEEEFLOAT} shall give the largest decimal floating point size (in bits) supported by the aliases. [Note: All variants support the same sizes. —end note] If none of these aliases are supported, the value shall be 0. The parameter to the alias shall be in the range 1 to \tcode{MAX_BITS_10IEEEFLOAT}.

\begin{itemdecl}
template<int bits> alias exact_10ieeefloat
\end{itemdecl}

\begin{itemdescr}
The alias \tcode{exact_10ieeefloat} refers to a built-in decimal floating-point type of exactly \tcode{bits} bits. If there are two types of the same size, it refers to the type that is closest to double in promotion order. The type must use IEEE representation.
\end{itemdescr}

\begin{itemdecl}
template<int bits> alias fast_10ieeefloat
\end{itemdecl}

\begin{itemdescr}
The alias \tcode{fast_10ieeefloat} refers to the fastest built-in decimal floating-point type of at least \tcode{bits} bits. If there are two types of the same size, it refers to the type that is closest to double in promotion order. The type must use IEEE representation.
\end{itemdescr}

\begin{itemdecl}
template<int bits> alias least_10ieeefloat
\end{itemdecl}

\begin{itemdescr}
The alias \tcode{least_10ieeefloat} refers to the smallest built-in decimal floating-point type of at least \tcode{bits} bits. If there are two types of the same size, it refers to the type that is closest to double in promotion order. The type must use IEEE representation.
\end{itemdescr}   
\end{addedblock}