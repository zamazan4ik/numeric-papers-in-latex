%!TEX root = ts.tex

\rSec0[foundations]{Foundations}

\rSec1[mission]{Mission}

\iffalse %% Uncomment for the TS
The mission of this specification is to make numbers easier to use.
\else
The mission of this document is to highlight inconsistencies among different numbers related proposals, provide a possible solution for them, and update the wordings to be more suitable for the LWG,.
\fi

\rSec1[intro.style]{Style of presentation}
Existing wording from the C++ working draft - included to provide context - is presented without decoration.

\begin{addedblock}
Entire clauses / subclauses / paragraphs incorporated from different numbers related proposals are presented in a distinct cyan
color.
\end{addedblock}


\addmodif{Wording to be added which is original to this document appears in blue.}

\remmodif{Wording which this document strikes is presented in red with strikethrough.}

\begin{modifcommentblock}
Notes and motivations for change are in brown.
\end{modifcommentblock}

\rSec1[intro.scope]{Scope}

This document provides numbers and their options, but does not provide algorithms or special functions.

\rSec1[intro.design]{Design principles}

Follow mathematical behavior where feasible.

\begin{itemize}
\item Make unbounded numbers very simple to use.
\item Make unavoidable overflow and rounding predictable and controllable.
\begin{itemize}
\item Overflow on any operation is unmanageable.
\item A rational number with fixed-size fields needs to round.
\end{itemize}
\item Avoid or hide aliasing effects.
\item Prefer compile-time errors to run-time errors.
\item Prefer safe defaults to effient defaults.
\end{itemize}

Provide types that match the taxonomic needs.

\begin{itemize}
\item E.g. fixed-point, extended floating-point.
\item Support construction of new types by library authors.
\begin{itemize}
\item Expose common implementation abstractions.
\end{itemize}	
\end{itemize}

Strive for efficiency.

\begin{itemize}
\item Use efficient function parameter passing.
\item Use efficient representations.
\item Give speed priority to dynamically common operations.
\item Provide composite operations when efficient.
\begin{itemize}
\item E.g. shift and add, multiply and add.
\end{itemize}
\item Identify opportunities for new hardware.
\begin{itemize}
\item E.g. rounding right shift.
\end{itemize}
\item Prefer run-time efficiency over compile-time efficiency.
\end{itemize}

Ease adoption and use.

\begin{itemize}
\item Provide a consistent vocabulary.
\item Enable value conversion between all applicable types.
\item Handle parameter aliasing within the implementations.
\end{itemize}

Ease extension.

\begin{itemize}
\item Provide a mechanism for conversion that does not require $n^{2}$ operations or coordination between independent developers.
\item Most parts of the implementation should need only C++, so provide a machine abstraction layer.
\item Expose sound "building-block" abstractions.
\end{itemize}

\rSec1[intro.taxonomy]{Number type taxonomy}

Types may be categorized by the representation constancy between argument and result.

\textbf{same}
The size of the type is the same between argument and result. Overflow is pervasive on all operations.
\begin{itemize}
\item invariant - All aspects of the representation are the same.
\item invariant - Different fields within the representation may have different sizes.
\end{itemize}
\textbf{adaptive}
The size of the result is statically known, but it may be different from (and generally larger than) the argument.
\begin{itemize}
\item limited - There is a maximum representation. If adaptation requires more than the maximum, the expression is ill-formed. Overflow in expressions is avoided until the maximum representation is reached.
\item unlimited - There is no a priori maximum representation. Overflow in expressions is avoided.
\end{itemize}
\textbf{dynamic}
All aspects of the representation are dynamic. Overflow is variables is avoided.

Types may also be categorized by the size specification, which is generally by the number of bits in the representation or by the number of digits required.

\begin{addedblock}
\rSec0[intro.operations]{Operations}

The various types will share many operations. To reduce suprise, these operations should have consistent names for consistent purposes.

\begin{libreqtab3}
    {Operations}
    {tab:operations}
    \\ \topline
    \lhdr{operator}  &
    \chdr{name}  &
    \rhdr{operation} \\ \capsep
    \endfirsthead
    \continuedcaption\\
    \hline
    \lhdr{operator}  &
    \chdr{name}  &
    \rhdr{operation} \\ \capsep
    \endhead

\textbf{arithmetic} & &
\\ \rowsep

- & neg & negate
\\ \rowsep

+ & add & add
\\ \rowsep

- & sub & subtract
\\ \rowsep 

* & mul & multiply
\\ \rowsep

& rdiv & divide with rounding
\\ \rowsep

& pdiv & divide with promotion
\\ \rowsep

& quorem & quotient and remainder
\\ \rowsep 

& mod & modulo
\\ \rowsep

& rem & remainder; different sign matching rules
\\ \rowsep

\textbf{scale/shift} & &
\\ \rowsep

& ssu & static scale up
\\ \rowsep

& ssd & static scale down with rounding
\\ \rowsep

& dsu & dynamic scale up
\\ \rowsep

& dsd & dynamic scale down with rounding
\\ \rowsep

& lsh & left shift
\\ \rowsep

& ash & arithmetic right shift
\\ \rowsep

& rsh & logical right shift
\\ \rowsep

& rtl & rotate left
\\ \rowsep

& rtr & rotate right
\\ \rowsep

\textbf{composite} & & 
\\ \rowsep

& add2 & add with two addends
\\ \rowsep

& sub2 & subtract with two subtrahends
\\ \rowsep

& lshadd & left shift and add
\\ \rowsep

& lshsub & left shift and sub
\\ \rowsep

& muladd & multiply and add
\\ \rowsep

& mulsub & multiply and subtract
\\ \rowsep

& muladd2 & multiply and add with two addends
\\ \rowsep

& mulsub2 & multiply and subtract with two subtrahends
\\ \rowsep

\textbf{bitwise} & &
\\ \rowsep

& not & 1's complement
\\ \rowsep

& and & and
\\ \rowsep

& xor & exclusive or
\\ \rowsep

& ior & inclusive or
\\ \rowsep

& dif & difference (a\&\~b)
\\ \rowsep
    
\end{libreqtab3}    

\end{addedblock}
